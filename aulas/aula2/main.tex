\documentclass[a4paper,11pt]{article}
%\documentclass[a4paper,draft]{article}
%\documentclass[a4paper]{article}
\usepackage[brazil]{babel}
\usepackage[utf8]{inputenc}
\usepackage{graphicx}
\usepackage{amsfonts}
\usepackage{charter} % nova fonte
\usepackage{indentfirst} % indenta o 1o parágrafo
\usepackage{minted}
\usepackage{enumerate}
\usepackage{color}
\usepackage{tikz}
\usetikzlibrary{matrix}
\usepackage{url}
\usepackage{multicol}
\usepackage{algpseudocode}

\usepackage{hyperref}
\hypersetup{
    colorlinks=true,
    linkcolor=red,
    filecolor=magenta,      
    urlcolor=red,
    pdftitle={Overleaf Example},
    pdfpagemode=FullScreen,
    }

\usepackage{amsmath}
\everymath{\displaystyle}
\setlength{\multicolsep}{4.0pt plus 2.0pt minus 1.5pt}% 50% of original values

% unidade usada para desenhar as figuras
\setlength{\unitlength}{1cm}
%\linethickness{1.5pt}

\newcommand{\remove}[1]{}%{#1}
\newtheorem{teorema}{Teorema}
\newenvironment{proof}[1][Prova]{\begin{trivlist}
\item[\hskip \labelsep {\bfseries #1}]}{\end{trivlist}}

% formatar margens quando utilizar 2 colunas
%\addtolength{\evensidemargin}{-1.5cm}\addtolength{\oddsidemargin}{-1.5cm}\addtolength{\textwidth}{2.7cm}
\addtolength{\evensidemargin}{-2cm}\addtolength{\oddsidemargin}{-2cm}\addtolength{\textwidth}{3.7cm}

\addtolength{\topmargin}{-4.0cm}
\addtolength{\textheight}{7.0cm}

%\setlength{\parskip}{\baselineskip} % aumenta espaço entre parágrafos
\setlength{\parskip}{0em}

\hyphenpenalty=100000
% convert probs_np_completos.gif probs_np_completos.eps
% ps2pdf -sPAPERSIZE=a4 np.ps

\begin{document}
\title{CK0226 - Programação \\
Professor Wladimir A. Tavares\\
Programação em C}
\date{}
\maketitle
%\tableofcontents

%\tiny
%\scriptsize
%\footnotesize
%\small
\normalsize
%\large
%\Large
%\LARGE
%\huge
%\Huge

\noindent

O objetivo da aula é aprender:

\begin{enumerate}
    \item Noções básica de comparações de algoritmo (Encontrar o Maior Elemento)
    \item a usar o comando while e if/else (busca Linear), (Jump Search passo 2) e  (Jump Search passo $\sqrt{n}$).
    \item Comportamento linear
    \item Comportamento raiz quadrada.
    \item Comportamento logarítmico.
    \item Desenvolvimento de funções \texttt{colocar\_maior\_ultimo}.
    \item Usar funções para resolver outros problemas (Ordena Colocando o maior por último)
\end{enumerate}

\newpage 


% \section*{Multiplicar todos os ímpares da lista por 2}

% Considere o problema de multiplicar todos os números ímpares da lista por 2.

% \begin{center}
% \begin{tikzpicture}
%     % Labels for L and indices
%     \node at (-7, 0) {$L$};
%     \node at (-5.5, 1.0) {1};
%     \node at (-4.5, 1.0) {2};
%     \node at (-3.5, 1.0) {3};
%     \node at (-2.5, 1.0) {4};
%     \node at (-1.5, 1.0) {5};
%     \node at (-0.5, 1.0) {6};
%     \node at (0.5, 1.0) {7};
%     \node at (1.5, 1.0) {8};
%     \node at (2.5, 1.0) {9};
%     \node at (3.5, 1.0) {10};
    
%     % Draw the boxes and numbers
%     \matrix[matrix of nodes, nodes in empty cells, 
%             nodes={draw, minimum size=1cm, anchor=center}] (m) at (-1,0) {
%         17 & 10 & 42 & 15 & 9 & 30 & 6 & 25 & 16 & 20  \\
%     };
% \end{tikzpicture}
% \end{center}

% \begin{minted}[frame=single,
%     obeytabs=true,
%     tabsize=4,
%     linenos]{C++}
% #include <stdio.h>
% int main(){
% 	int i;
% 	int L[6] = {17,10,42,15,9,30,6,25,16,20}; 
% 	for(i = 0; i < 6; i++){
% 		if(L[i]%2==1)
% 			L[i] = L[i]*2;
% 	}
% 	imprime_lista(L, 6);
% }
% /*
% [ 2 6 4 10 14 8]
% */
% \end{minted}

% A ideia por trás do nosso algoritmo é processar cada elemento da lista, um por vez, e garantir que, ao final, toda a lista esteja composta apenas por números pares. 

% A cada passo do laço, garantimos que todos os números que já vimos (isto é, todos os elementos da sublista $L[0..i-1]$) já foram modificados.

% Podemos descrever essa propriedade da seguinte maneira:
% \begin{quote}
%     No início de cada iteração (dedo na posição) i, todos os elementos da lista $L[0..i-1]$ são pares.
% \end{quote}


% Quando o dedo está na posição $i$, verificamos $L[i]$: 
% \begin{itemize}
%     \item Se $L[i]$ já é par, não precisamos fazer nada e a propriedade continua verdadeira.
%     \item Se $L[i]$ for ímpar, multiplicamos $L[i]$ por 2, transformando-o em par.
% \end{itemize}

% Na última iteração, temos que $i = n$ e a propriedade que estamos mantendo garante que todos os elementos da sublista $L[0..n-1]$) são pares.

% O número de instruções realizadas pelo programa acima é proporcional a $n$. Logo, dizemos que o programa tem comportamento linear com relação ao tempo de execução.

\section*{Encontrando o maior elemento}

Quando um algoritmo é bom? Quando um algoritmo é ruim?

\begin{minted}[frame=single,
    obeytabs=true,
    tabsize=4,
    linenos]{C++}
int maior_elemento1(int L[], int n){
    for(int i = 0; i < n; i++){
        int maior_todos = 1;
        for(int j = 0; j < n; j++){
            if(L[j] > L[i]){
                maior_todos = 0;
                break;
            }
        }
        if(maior_todos == 1){
            return L[i];
        }
    }
}
\end{minted}

\begin{minted}[frame=single,
    obeytabs=true,
    tabsize=4,
    linenos]{C++}
int maior_elemento2(int L[], int n){
    int M = L[0];
    for(int i = 1; i < n; i++){
        if(L[i] > M){
            M = L[i];
        }
    }
    return M;
}

\end{minted}

As duas funções acima encontram o maior elemento numa lista de números. No caso acima, o algoritmo \texttt{maior\_elemento2} é melhor que \texttt{maior\_elemento1} (Por quê?). 

Podemos tentar olhar para o número de comparações para tentar comparar os dois algoritmos. O algoritmo 
\texttt{maior\_elemento2} realiza $n^2$ comparações e 
\texttt{maior\_elemento1} realiza $n-1$ comparações.

\newpage 




% \section*{Varredura}

% A ideia por trás da varredura do vetor é garantir que, no início de cada iteração $i$, o maior elemento da sublista $L[0..i]$ esteja na posição $L[i]$. Para alcançar isso, verificamos se o próximo elemento $L[i+1]$ é menor que $L[i]$. Se $L[i]$ for maior que $L[i+1]$, significa que esses dois elementos estão fora de ordem, e uma troca é necessária para "empurrar" o maior deles para a posição correta ao final da sublista $L[0..i+1]$. 

% Essa verificação e troca são repetidas para cada par de elementos adjacentes, empurrando progressivamente os maiores elementos para as últimas posições da lista até que ela esteja totalmente ordenada.

% \begin{minted}[frame=single,
%     obeytabs=true,
%     tabsize=4,
%     linenos]{C++}
% #include <stdio.h>
% #define N 6
% int L[N] = {7,9,1,2,3,5}; 
	
% int main(){
% 	int i, temp;
% 	// Garante que ao final de cada iteração, o maior elemento de L[0..i] está em L[i]
% 	for(i=0;i<N-1;i++){
% 		if(L[i] > L[i+1]){
% 			temp = L[i];
% 			L[i] = L[i+1];
% 			L[i+1] = temp;
% 		}
% 	}
	
% 	printf("[");
% 	for(i=0; i<N; i++){
% 		printf(" %d", L[i]);
% 	}
% 	printf("]\n");	
	
% }
% \end{minted}




% \section*{Ordena varredura}

% \section*{Checar se um vetor é crescente usando for}

% \section*{Checar se um vetor é crescente usando while}

% \section*{Inverte o vetor}

% \section*{Pares no começo e ímpares no final}

% \section*{Mover o maior elemento com o último sem alterar a posição dos demais}


% \begin{minted}[frame=single,
%     obeytabs=true,
%     tabsize=4,
%     linenos]{C++}
% #include <stdio.h>
% #include "prog.h"
% int main(){
% 	int L[6] = {7,9,1,2,3,5}; 
% 	int i, posM, M, n, aux;
% 	n = 6;
% 	//Encontrar a posição do maior
% 	M = L[0];
% 	posM = 0;
% 	for(i = 0; i < 6; i++){
% 		if(L[i] > M){
% 			M = L[i];
% 			posM = i;
% 		}
% 	}
	
% 	for(i = posM; i < n-1; i++){
% 		L[i] = L[i+1];
% 	}
% 	L[n-1] = M;
% 	imprime_lista(L, n);
% }
% \end{minted}

\newpage 

\section*{Busca Linear}

Nesse problema, recebemos um lista de inteiros $L$ ordenada em ordem crescente e um inteiro $k$, queremos saber se o inteiro $k$ é um elemento da lista de inteiros $L$.

O algoritmo de busca linear consiste em avançar a variável $i$ em uma unidade até encontrar um elemento $L[i] \geq k$ ou $i = n$. Neste caso, podemos analisar os seguintes casos:

\begin{enumerate}
    \item $L[i] == k$, o elemento está presente na lista.
    \item $L[i] > k$, o elemento não está presente na lista.
    \item $i == n$, o elemento não está presente na lista.
\end{enumerate}

\begin{minted}[frame=single,
    obeytabs=true,
    tabsize=4,
    linenos]{C++}
#include <stdio.h>
#define N 10
int L[N] = {2,5,6,9,11,15,18,20,21,25};
int k = 14;

int main(){
	int i;
	i = 0;
	while(i<N && L[i] < k) { 
		printf("%d < %d\n", L[i], k);
		i++;
	}
	
	if(i==N) printf("nao achei\n");
	else if(L[i]==k) printf("achei\n");
	else printf("nao achei\n");
	/*
	if(i==N || L[i]>k)
		printf("nao achei");
	else
		printf("achei");
	*/
}
\end{minted}

O programa acima realiza aproximadamente $n$ comparações. Neste caso, dizemos que o crescimento do tempo de execução tem um comportamento linear.


\newpage 


\section*{Jump Search passo 2}

Nesse problema, recebemos um lista de inteiros $L$ ordenada em ordem crescente e um inteiro $k$, queremos saber se o inteiro $k$ é um elemento da lista de inteiros $L$.

O algoritmo de busca linear consiste em avançar a variável $i$ em duas unidade até encontrar um elemento $L[i] \geq k$ ou $i \geq n n$. Neste caso, podemos analisar os seguintes casos:


\begin{enumerate}
    \item $i == n+1$, o último elemento foi testado e o elemento k não está presente.
    \item $i == n$, o último elemento não foi testado.
    \item $i < n \mbox{ and } L[i] \geq k$, precisamos testar $L[i]==k$ e $L[i-1]==k$   
\end{enumerate}


\begin{minted}[frame=single,
    obeytabs=true,
    tabsize=4,
    linenos]{C++}
#include <stdio.h>
#define N 10
int L[N] = {2,5,6,9,11,15,18,20,21,25};
int k = 25;

int main(){
	int i;
	i = 0;
	while(i<N && L[i] < k) { 
		printf("i %d : %d < %d\n", i, L[i], k);
		i = i + 2;
	}
	printf("i = %d\n", i);
	/*
	//testou o ultimo
	if(i == N+1)
		printf("nao achei\n");
	else if(i == N) { //pulou o ultimo
		if(L[i-1] == k) printf("achei\n");
		else printf("nao achei");
	}else{ // i < N
		if(L[i] == k) printf("achei");
		else if(i > 0 && L[i-1] == k)             
                 printf("achei\n");
		else printf("nao achei");
	}
	*/
	if(i>=N){
		if(i-1<N && L[i-1]==k) printf("achei");
		else printf("nao achei\n");
	}else{
		if(L[i]==k || (i > 0 && L[i-1] == k))
			printf("achei\n");
		else
			printf("nao achei\n");
	}
	
}
\end{minted}

O programa acima realiza aproximadamente $\frac{n}{2}$ comparações. Neste caso, dizemos que o crescimento do tempo de execução tem um comportamento linear.


\newpage 

\section*{Jump Search passo $\sqrt{n}$}

No algoritmo Jump Search, o número de comparações realizada será aproximadamente ao número de blocos + tamanho do bloco. Podemos tentar encontrar o melhor tamanho de bloco, analisando os seguintes casos para N = 10:

\begin{tabular}{lll}
pulo & número de blocos & tamanho do bloco\\
2    & $\frac{n}{2}$      & 2\\
3    & $\frac{n}{3}$      & 3\\
\end{tabular}

Quando escolhemos o tamanho do pulo = $\sqrt{n}$ temos que o número de blocos e o tamanho do bloco tem quase o mesmo tamanho.

\begin{minted}[frame=single,
    obeytabs=true,
    tabsize=4,
    linenos]{C++}
#include <stdio.h>
#include <math.h>
#define N 9
int L[N] = {2,5,6,9,11,15,18,20,21};
int k = 11;
int pulo = sqrt(N);
int main(){
	int i;
	i = 0;
	while(i<N && L[i] < k) { 
		printf("i %d : %d < %d\n", i, L[i], k);
		i = i + pulo;
	}
	printf("i = %d\n", i);
	if(i >= N){
		i = N-1;
		while(i >= 0 && L[i] > k) { 
			printf("i %d : %d > %d\n", i, L[i], k);
			i = i-1;
		}	
		if( i < 0 || L[i] < k) printf("nao achei\n");
		else printf("achei");
	}else{
		while(i >= 0 && L[i] > k) { 
			printf("i %d : %d > %d\n", i, L[i], k);
			i--;
		}
		if( i < 0 || L[i] < k) printf("nao achei\n");
		else printf("achei\n");
	}	
}
\end{minted}

Note que para encontrar o valor 20, o programa vai imprimir o seguinte relatório:
\begin{minted}[frame=single,
    obeytabs=true,
    tabsize=4,
    linenos]{C++}
i 0 : 2 < 20
i 3 : 9 < 20
i 6 : 18 < 20
i = 9
i 8 : 21 > 20
achei
\end{minted}

Note a variável $i$ pula até 3 blocos e sai do lista chegando ao valor $i = 9$. Neste caso, ele precisa analisar o último bloco da esquerda para direita pulando sequencialmente os números maiores que o número buscado. No caso do número 20, o laço termina com a variável $i$ apontando para o número 20.

\newpage 

\section*{Busca Binária}

Na busca binária, precisamos jogar fora sempre a metade do problema. Quando escolhemos um elemento na posição do meio, pode acontecer três coisas:

\begin{itemize}
\item se $L[meio] == k$, então a busca pode ser interrompida.
\item se $L[meio] > k$ então sabemos que o elemento buscado está na primeira metade ($L[inicio..meio-1]$).
\item se $L[meio] < k$ então sabemos que o elemento buscado está na segunda metade ($L[meio+1..fim]$).

\end{itemize}

\begin{minted}[frame=single,
    obeytabs=true,
    tabsize=4,
    linenos]{C++}
#include <stdio.h>
#include <math.h>
#define N 9
int L[N] = {2,5,6,9,11,15,18,20,21};
int k = 20;
int main(){
	int inicio, fim, meio, encontrei;
	inicio = 0;
	fim = N-1;
	encontrei = 0;
	while(inicio<fim){
		printf("L[%d...%d]\n", inicio, fim);
		meio = (inicio+fim)/2;
		if(L[meio] == k){
			encontrei = 1;
			break;
		}else if(L[meio] > k){
			fim = meio-1;
		}else{
			inicio = meio+1;
		}
	}	
}
\end{minted}

\newpage 

\section*{Trocar o maior elemento com o último}

Podemos resolver o problema em duas etapas:

\begin{enumerate}
    \item Encontrar a posição do maior.
    \item Trocar o maior com a última posição
\end{enumerate}

A ideia do algoritmo é utilizar a variável $M$ para guardar o maior encontrado visto até agora e $posM$ guarda a posição do maior encontrado visto até agora. Ao final do primeiro laço, encontramos o maior elemento e a sua posição. Em seguida, realizamos a troca desse elemento com o último.



\begin{minted}[frame=single,
    obeytabs=true,
    tabsize=4,
    linenos]{C++}
#include <stdio.h>
#include "prog.h"
int main(){
	int L[6] = {7,9,1,2,3,5}; 
	int i, posM, M, n, aux;
	n = 6;
	//Encontrar a posição do maior
	M = L[0];
	posM = 0;
	for(i = 0; i < 6; i++){
		if(L[i] > M){
			M = L[i];
			posM = i;
		}
	}
	//Trocar o maior com o último
	aux = L[posM];
	L[posM] = L[n-1];
	L[n-1] = aux;
	
	imprime_lista(L, 6);
}
\end{minted}

\newpage 


\section*{Coloca\_maior\_ultimo}

Imagine que queremos realizar essa tarefa muitas vezes para diferentes faixas da listas. A ideia agora é definir uma função para isso precisamos definir um nome e quais são os parâmetros de entrada para a função.

\begin{minted}[frame=single,
    obeytabs=true,
    tabsize=4,
    linenos]{C++}
#include <stdio.h>
#define N 6
int L[6] = {7,9,1,2,3,5}; 

void coloca_maior_ultimo(int L[], int n){
	int i, M, posM, aux;
	
	M = L[0];
	posM = 0;
	for(i = 0; i < n; i++){
		if(L[i] > M){
			M = L[i];
			posM = i;
		}
	}
	//Trocar o maior com o último
	aux = L[posM];
	L[posM] = L[n-1];
	L[n-1] = aux;

}


int main(){

                           //[7,9,1,2,3,5]   
coloca_maior_ultimo(L, 6); //[7,5,1,2,3|,9]
coloca_maior_ultimo(L, 5); //[3,5,1,2|,7,9]
coloca_maior_ultimo(L, 4); //[3,2,1|,5,7,9]
coloca_maior_ultimo(L, 3); //[1,2|,3,5,7,9]
coloca_maior_ultimo(L, 2); //[1,|2,3,5,7,9]
 
		
}
\end{minted}

O número de instruções realizadas pela função \texttt{colocar\_maior\_ultimo} é proporcional a $n$. Logo, dizemos que a função tem comportamento linear com relação ao tempo de execução.


\newpage

\section*{ Ordena colocando o maior por último}

A ideia do método de ordenação de colocar o maior na última posição consiste em cada passo do laço, o maior é colocado na última posição do vetor de tamanho i. O problema pode ser reduzido em 1 unidade. Note que o processo continua até que o vetor fique o com o tamanho igual a 1. 

\begin{minted}[frame=single,
    obeytabs=true,
    tabsize=4,
    linenos]{C++}
#include <stdio.h>
#define N 6
int L[6] = {7,9,1,2,3,5}; 

void coloca_maior_ultimo(int L[], int n){
	int i, M, posM, aux;
	
	M = L[0];
	posM = 0;
	for(i = 0; i < n; i++){
		if(L[i] > M){
			M = L[i];
			posM = i;
		}
	}
	//Trocar o maior com o último
	aux = L[posM];
	L[posM] = L[n-1];
	L[n-1] = aux;

}


int main(){

    int i;
    for(i = N; i > 1; i++){
        coloca_maior_ultimo(L, i);
    }
		
}
\end{minted}

O número de instruções realizadas pelo programa acima é proporcional a $n^2$. Logo, dizemos que o programa tem comportamento quadrático com relação ao tempo de execução.

\end{document}	